\chapter{Conclusão}

A quantidade de ácido acetilsalicílico foi determinada nos seguintes comprimidos: Aspirina\R, 
Melhoral\R, genérico e AAS Infantil.

Foram realizados dois experimentos. Para o primeiro experimento, todas as massas calculadas
encontraram-se fora do intervalo aceitável de $\pm$ 5\%, segundo a Farmacopeia Brasileira, o qual
foi adotado neste experimento. O resultado mais próximo do real foi o do genérico, com um erro
relativo de de $7,148\%$ e houve uma maior descrepância nos dados da Aspirina\R, com erro relativo
de $20,104\%$.

No segundo experimento, todos os resultados encontraram-se dentro do intervalo aceitável, exceto o
genérico, o qual aprensentou erro relativo de $6,643\%$. Já o AAS Infantil foi o que apresentou mais
proximidade em relação ao real, tendo um erro relativo de $3,537\%$.
