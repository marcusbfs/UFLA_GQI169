\chapter{Discussão}

Para o experimento 1 todas as massas obtidas foram maiores que as especificadas na bula, sendo que
para a Aspirina\R \; a massa obtida foi 100\% $m/m$, valor considerado muito elevado, sendo que o
erro pode ter ocorrido durante a titulação, por erro do operador (na pesagem dos comprimidos, na
transferência de solução entre vidrarias ou na titulação) ou pelo pouco volume de amostra a ser
titulada (10 mL), favorecendo um erro maior no caso de volume do titulante em excesso. Para os
medicamentos Genérico, Melhoral\R e AAS infantil as massas de AAS encontradas foram maiores que
$\pm$ 5\% do valor da massa de ácido acetilsalicílico especificada na bula.

Considerando esses resultados inconsistentes, foi feito um novo experimento, com mudanças nos
procedimentos, visando melhorar os resultados.

 No segundo experimento, para a Aspirina\R, Melhoral\R e AAS Infantil, as massas de AAS encontradas
 estavam menores que as especificadas nas bulas, mas estavam dentro de um intervalo de $\pm$ 5\% das
 massas especificadas em bula(entre 95\% e 105\%). Para o Genérico, a massa estava menor que a
 especificada em bula e ainda fora do intervalo de $\pm$ 5\% da massa especificada pelo fabricante.

 Segundo a Farmacopeia Brasileira \cite{Farmacopeia2010} comprimidos de ácido acetilsalicílico devem
 conter, no mínimo, 95,0\% e, no máximo, 105,0\% da quantidade declarada de \ce{C9H8O4}.

 Considerando apenas o segundo experimento como valores consideráveis, os valores de massa
 encontrados para a Aspirina\R, o Melhoral\R e o AAS Infantil encontram-se dentro da faixa de valor
 aceitável pela Farmacopeia Brasileira. A massa encontrada para o Genérico encontra-se fora do
 intervalo aceito pela Farmacopeia Brasileira (entre 95\% e 105\% da massa de AAS especificada).

 \begin{table}[H]\label{err_exp1}
     \centering
\begin{tabular}{c c}
    \toprule
    \textbf{Medicamento} & \textbf{Erro relativo (\%)} \\
    \midrule
    Aspirina\R & 20,10 \\
    Genérico & 7,15 \\
    Melhoral\R & 7,32 \\
    AAS Infantil & 18,18\\
    \bottomrule
\end{tabular}
\caption{Erros relativos experimento 1}
 \end{table}

 \begin{table}[H]\label{err_exp2}
     \centering
\begin{tabular}{c c}
    \toprule
    \textbf{Medicamento} & \textbf{Erro relativo (\%)} \\
    \midrule
    Aspirina\R & 4,90 \\
    Genérico & 6,65 \\
    Melhoral\R & 4,60 \\
    AAS Infantil & 3,54 \\
    \bottomrule
\end{tabular}
\caption{Erros relativos experimento 2}
 \end{table}
