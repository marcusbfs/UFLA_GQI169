\chapter{Introdução}\label{intro}
\section{Justificativa}\label{sub:just}    

O ácido acetilsalicílico é, possivelmente, o medicamento mais conhecido e vendido no mundo todo
Desse modo, a análise dos comprimidos comercializados é de interesse dos estudantes da Engenharia Química,
uma vez que proporciona aprendizado a cerca da indústria farmacêutica, padrões de qualidade, além de
práticas laboratoriais da teoria de química analítica e orgânica.

\section{Objetivos}\label{sub:Objetivos}
Determinar o teor de ácido acetilsalicílico contido nos comprimidos comercializados, analisando
medicamentos de referência (Aspirina\R), genéricos e similares.

\section{Referencial Teórico}\label{sub:reft}

\subsection{Histórico}\label{sub:Histórico}

O ácido acetilsalicílico (AAS) é um dos medicamentos mais populares no mundo, sendo conhecido por
seu nome comercial: Aspirina\R. Ele é um analgésico, anti-inflamatório e antipirético produto de uma
reação que tem como um dos reagentes o ácido salicílico, extraído do salgueiro (\textit{Salix
alba}).  

O ácido acetilsalicílico pode ser identificado também pela sua fórmula química \ce{C9H8O4} e seu
nome IUPAC ácido 2-acetóxibenzóico. A respeito de propriedades físico-químicas, o AAS é um pó
cristalino branco, inodoro, solúvel em álcool e éter e pouco solúvel em água, possui ponto de fusão
135ºC e ponto de ebulição a 140ºC. Ecologicamente, é facilmente biodegradado em estações de
tratamento de água e não bioacumula.~\cite{teves}

A utilização do ácido salicílico, no alívio de dores já existe há séculos, isso porque essa
substância está contida em diversas plantas consideradas medicinais. “Uma coleção de anotações
datadas de cerca de 1500 a.C., conhecidas como papiros de Ebers, já recomendava o uso da infusão de
folhas de murta para o alívio de dores reumáticas”~\cite{aspirinabayer}. Posteriormente, no século V
a.C., Hipócrates, o pai da medicina moderna, registrou que o pó da casca do salgueiro era capaz de
amenizar dores e febre. Somente em 1860 a substância encontrada na casca do salgueiro foi isolada em
laboratório e recebeu o nome de salicilato, que designa um “grupo de fármacos que atuam devido ao
seu conteúdo de ácido salicílico”.~\cite{Goodman2005}

Já o descobrimento do ácido acetilsalicílico ocorreu mais tarde, “quando o químico alemão Felix
Hoffman pesquisava um medicamento para ser usado no tratamento da artrite, doença de seu pai. O
objetivo dele era encontrar uma droga para substituir o salicilato de sódio, medicamento usado
naquela época, mas que exigia grandes doses diárias e provocava irritação e fortes dores estomacais
nos pacientes”.~\cite{massabni2006}

Em 1897, Hoffman, que trabalhava na companhia Bayer da Alemanha, preparou o ácido acetilsalicílico
combinando o ácido salicílico com acetato. A reação resultou numa substância mais vantajosa do que o
salicilato de sódio, em questão de eficiência e efeitos colaterais. Tal droga recebeu,
posteriormente, o nome de Aspirina\R e se tornou o primeiro fármaco sintetizado em laboratório.  

A Aspirina\R teve sua patente concedida em 1899 e começou a ser comercializada.  Ainda que
inicialmente os superiores de Hoffman achassem que o medicamento fracassaria, o mesmo tornou-se
sucesso de vendas e inclusive destacou-se como medicamento mais utilizado no tratamento da artrite.
Inicialmente, a Aspirina\R era vendida na forma de pó, entretanto, em 1900, ela tornou-se o primeiro
medicamento no mundo a ser vendido em doses padronizadas, que eram comprimidos com 500mg de ácido
acetilsalicílico. “A formulação em comprimidos tinha três vantagens principais: assegurar que cada
comprimido tivesse uma dose exata do ingrediente ativo, acabar com as falsificações dos produtos e
reduzir os custos de produção".~\cite{aspirinabayer}

Décadas depois, John Vane, Professor de Farmacologia do London Royal College for Surgeons, “observou
que alguns tipos de ferimento eram acompanhados da liberação em nosso corpo de substâncias chamadas
de prostaglandinas. Ele também percebeu que dois grupos delas provocavam febre e vermelhidão no
local do ferimento (sinais de inflamação). Vane e colaboradores descobriram que a Aspirina\R
bloqueava a síntese de prostaglandinas, evitando a formação de plaquetas, que depois se
transformavam em coágulos de sangue no corpo humano. Esses coágulos eram responsáveis pelo bloqueio
do fluxo de sangue para o coração, resultando no ataque cardíaco. Assim, a Aspirina\R evita a
formação de coágulos e, portanto, pode impedir o infarto do miocárdio”~\cite{massabni2006}. Em 1971,
Vane publicou no jornal “Nature” seus estudos sobre o mecanismo de ação do ácido acetilsalicílico,
as descobertas feitas por ele lhe renderam o Prêmio Nobel de Medicina de 1982. 

Nos últimos 30 anos, pesquisa foram feitas com diferentes grupos de pessoas que apresentavam
problemas cardiovasculares, cerebrovasculares e também pessoas sadias. Os resultados mostraram que a
Aspirina\R teve um grande impacto no tratamento e prevenção das doenças cardiovasculares.  Além
disso, pelo fato de ser anticoagulante, há trabalhos que mostram que a Aspirina\R reduz o risco de
trombose e derrame cerebral. No entanto, algumas pessoas podem apresentar efeitos colaterais como
dores estomacais, diarréias, náuseas, sangramento e hemorragia interna. Não é recomendado a sua
utilização para quem possui problemas renais ou gástricos.

\subsection{Síntese de ácido acetilsalicílico}

A síntese da Aspirina\R é dada através de uma reação de acetilação do ácido salicílico, que é um
composto aromático bifuncional, possuindo os grupos fenol e ácido carboxílico. A acetilação ou
etanoilação é o processo de introdução do grupo acetila (ou etanoila) em um composto orgânico.

\begin{figure}[H]
\begin{center}
    \includegraphics[width=.4\textwidth]{figuras/im1.png}
\end{center}
\caption{Grupo acetila ligado a uma cadeia carbônica.}\label{fig:im1}
\end{figure}

O radical acetila possui o grupo metila (\ce{CH3}-) conectado por uma ligação simples a um
carbonila. O carbono do grupo carbonila possui um único elétron livre, com o qual forma uma ligação
com o radical R da molécula.

O ácido sulfúrico irá agir na reação como um catalisador, sendo assim ele irá se ionizar liberando
um \ce{H+} que se ligará em um dos oxigênios presentes na molécula de anidrido acético. Quando isso
ocorre, o oxigênio que recebeu o hidrogênio deixa de fazer ligação dupla com o carbono e
consequentemente o carbono fica com três ligações, tornando-se um carbocátion. Como o carbono está
instável, ele busca a sua estabilidade na molécula de ácido salicílico com a qual ele está reagindo.
Portanto, o carbono liga-se na hidroxila do ácido salicílico, mas é importante ressaltar que a
hidroxila que ele irá se ligar é a hidroxila ligada diretamente ao anel benzênico, devido a forças
de atração. Na química orgânica, as ligações de acetilação tem preferências para se ligarem em
hidroxilas em posições orto e para, como é o caso do ácido salicílico (posição orto da hidroxila). A
formação da nova molécula de ácido salicílico com o carbocátion do anidrido acético possui um
oxigênio que contém três ligações, o qual se rearranja na molécula se desprotonando para
estabilizar, ou seja, o hidrogênio ligado a ele irá se ligar no oxigênio que faz parte do anidrido
acético, formando assim um ácido acético que irá se desprender da molécula. A ligação de elétrons
que estava ligada com a parte do ácido acético se direciona para o respectivo carbono que estava
fazendo tal ligação, mas antes disso o hidrogênio da hidroxila que estava ligada a esse carbono
volta para o catalisador (formando novamente o \ce{H2SO4}). Como o carbono recebeu os pares de
elétrons e o oxigênio da hidroxila perdeu o seu hidrogênio, acontece uma dupla ligação entre o
carbono e o oxigênio. Dando origem a molécula de ácido acetilsalicílico.

A reação entre um anidrido e um álcool (ou hidroxiácido) gera um ácido carboxílico e um éster. A
síntese do ácido acetilsalicílico realiza esta reação, um hidroxiácido (ácido salicílico) e um
anidrido (anidrido acético).

% Remover '\newpage' caso a imagem a baixo altera sua posição
\newpage

\begin{figure}[H]
\begin{center}
    \includegraphics[width=1.1\textwidth]{figuras/im2_sintese.png}
\end{center}
\caption{Reação de síntese do ácido acetilsalicílico}\label{fig:im2}%
\end{figure}

Segundo BRUICE (2006), a reação de carboxilação de Kolbe-Schmitt ficou conhecida como a primeira
etapa da síntese industrial de aspirina. Ela consiste na reação do íon fenolato com o dióxido de
carbono, sob pressão, formando o ácido salicílico. Este por sua vez, ao sofrer acetilação com o
ácido acético forma o ácido acetilsalicílico.~\cite{Bruice2006}

Há diversas maneiras de sintetizar o ácido acetilsalicílico, sendo que a matéria prima para a
sintetização é o ácido salicílico e as variações ocorrem com os agentes de acetilação e os
catalisadores.

Em relação aos agentes de acetilação pode-se fazer uso também do cloreto de acetila e do ácido
acético glacial (com redução da água formada na reação) na síntese do ácido acetilsalicílico.  No
entanto, o procedimento envolvendo o ácido acético glacial requer longo tempo de aquecimento, apesar
de apresentar um custo inferior. Já o cloreto de acetila não é recomendado porque ele é muito
reativo. Ele se hidrolisa facilmente com a umidade do ar e em temperatura ambiente. O anidrido
acético é o agente de acetilação mais visionado nas reações de laboratório, porque sua velocidade de
hidrólise é suficientemente lenta para permitir que a acetilação seja realizada com maior
rendimento.~\cite{PERUCH2013}

No trabalho foi explicitado a acetilação do ácido salicílico com o anidrido acético na produção de
ácido acetilsalicílico, porque essa síntese possui características comerciais que são mais
favoráveis às indústrias químicas devido a sua eficiência e o seu baixo custo. 

Dentre a abundância de formas na produção de ácido acetilsalicílico a seguir estão apresentadas
algumas delas.

\begin{figure}[H]
\begin{center}
    \includegraphics[scale=0.83]{figuras/abiquim.png}
\end{center}
\caption{Cadeia industrial da produção de Aspirina\R. (Fonte: ABIQUIM, 2009)}
\label{fig:abiquim}
\end{figure}

Esse esquema foi realizado pela  Associação Brasileria de Indústrias Químicas (ABIQUIM), no qual
envolveu a cadeia industrial da produção de aspirina.~\cite{abiquim}

\subsection{Mecanismo de ação da Aspirina}

O ácido acetilsalicílico é um anti-inflamatório não esteroide (AINEs) ou não hormonal (AINHs), tendo
como propriedades analgésica, antipirética e anti-inflamatória. Além disso, possui um efeito
inibitório sobre as plaquetas no sangue e apresenta uma redução de 40\% no infarto do miocárdio
fatal e não fatal.~\cite{Palomo2008}

É usado para o alívio da dor e de quadros febris, tais como resfriados e gripes, para controle da
temperatura e alívio das dores musculares e das articulações. Também é usado nos distúrbios
inflamatórios agudos e crônicos, tais como artrite reumatoide, osteoartrite e espondilite
anquilosante.~\cite{bulaaspirina}

A atuação do ácido acetilsalicílico no organismo é de coibir a ação da ciclo-oxigenase. Essa ação
utiliza as enzimas COX-1 e COX-2, as quais transformam o ácido araquidônico em prostaglandinas, que
são responsáveis por induzir a dor e a febre. Esse ácido araquidônico é oriundo da ação enzimática
da Fosfolipase A2 sobre os fosfolipídios presentes nas membranas celulares. 

A plaqueta, importantíssima no processo de coagulação, é rica em COX-1 que, quando necessário,
produz prostaglandinas que são convertidas em Tramboxane A2 levando à formação de um coágulo. O
processo de coagulação se finda quando a Aspirina\R age na inflamação. Com o cessar da ação da
Aspirina\R no organismo, o processo de coagulação é retomado. Porém, uma vez que as plaquetas não
possuem núcleos elas não podem sintetizar uma nova COX para substituir a que foi inativa pela
Aspirina\R. Daí a ação anti-agregante plaquetária, que é um efeito irreversível. Entretanto, essa
ação irreversível não ocorre no endotélio vascular, pois este tem a capacidade de sintetizar as
novas moléculas. 

Devido a esse efeito inibitório da agregação plaquetária, o ácido acetilsalicílico pode aumentar a
tendência de sangramentos durante e após intervenções cirúrgicas (inclusive cirurgias de pequeno
porte, como por exemplo, extrações dentárias).~\cite{bulaaspirina}

Ademais, ao inibir a COX-2 ocorrerá o efeito anti-inflamatório, que diminuirá a inflamação vascular
no sítio da placa ateromatosa e esta, por sua vez, reduz a infiltração de células mononucleares na
placa de ateroma.~\cite{Grassi2012}

Algumas das funções das prostaglandinas estão na produção do muco estomacal, na coagulação sanguínea
e na manutenção da taxa de filtração glomerular nos rins. Como o fármaco (ácido acetilsalicílico)
inibe as enzimas COXs não há a produção das prostaglandinas, tendo como consequências alguns efeitos
no organismo.  Tais como, a úlcera péptica e uma posterior hemorragia digestiva, dependendo da
gravidade do quadro; a insuficiência renal aguda e problemas na coagulação. 

Após a medicação por via oral, o ácido acetilsalicílico é absorvido de forma rápida e completa no
trato gastrintestinal. Depois de absorvido e durante a sua absorção, o ácido acetilsalicílico é
convertido em ácido salicílico, que é o seu principal metabólito ativo. A quantidade máxima de
fármaco na corrente sanguínea do ácido acetilsalicílico é atingida após 10 a 20 minutos e a do ácido
salicílico após 0,3 a 2 horas. 

O ácido acetilsalicílico e o ácido salicílico possuem uma grande afinidade com as proteínas
plasmáticas, logo eles ligam-se extensivamente com tais proteínas e são distribuídos por todo o
organismo rapidamente.  O ácido salicílico passa para o leite materno e atravessa a placenta.

O ácido salicílico é eliminado principalmente por metabolismo hepático. Seus metabólitos incluem o
ácido salicilúrico, o glicuronídeo salicílico fenólico, o glicuronídeo salicilacílico, o ácido
gentísico e o ácido gentisúrico.

O metabolismo é restringido pela capacidade das enzimas hepáticas, sendo assim a cinética da
eliminação do ácido salicílico é dose-dependente. “A meia-vida de eliminação varia de 2 a 3 horas
após doses baixas até cerca de 15 horas com doses altas”~\cite{bulaaspirina}. O ácido salicílico e
seus metabólicos são excretados predominantemente por via renal. 

É importante frisar que esses anti-inflamatórios não esteroides (AINEs), do qual o ácido
acetilsalicílico faz parte, apenas inibem a produção de prostaglandinas na inflamação, mas eles não
cessam a mesma.

\subsection{Medicamentos de referência, genéricos e similares}\label{refgensim}

No mercado farmacêutico, medicamentos com o mesmo princípio ativo podem surgir com diferentes marcas
ou classificações. No caso do ácido acetilsalicílico, algumas são associadas a outras substâncias
como cafeína ou vitamina C.Outras possuem revestimentos de cápsula que buscam diminuir a agressão ao
sistema digestivo.  \cite{prade2006}

Com essas diferenças, os remédios são classificados como referência, genérico e similar. Os
medicamentos de referência, que também são conhecidos como “de marca”, são remédios que possuem
eficiência terapêutica, com qualidade e seguranças comprovadas cientificamente. São registrados
juntamente à Agência Nacional de substância Sanitária. Geralmente são medicamentos com novos
princípios ativos o que trazem novidades para o tratamento da doença

Com essas diferenças, os remédios são classificados como referência, genérico e similar. Os
medicamentos de referência, que também são conhecidos como “de marca”, são remédios que possuem
eficiência terapêutica, com qualidade e seguranças comprovadas cientificamente. São registrados
juntamente à Agência Nacional de substância Sanitária. Geralmente são medicamentos com novos
princípios ativos o que trazem novidades para o tratamento da doença.

Já os similares, além de possuir o mesmo princípio ativo do medicamento de referência, possui uma
identificação de nome comercial ou marca. A diferença é apresentada em alguns aspectos como
embalagem, rótulo, tamanho, validade e forma do produto. De acordo com a regulamentação da ANVISA,
dado uma prescrição médica os medicamentos de referência não podem ser substituídos por similares.

Para a possibilidade de troca desses medicamentos, ou seja, serem intercambiáveis, eles devem
apresentar um dos três testes: bioequivalência, biodisponibilidade e bioisenção. O estudo de
bioequivalência é sempre realizado entre o medicamento de referência e o estudado. Estas análises
tem o intuito de reafirmar a igualdade entre os produtos, proporcionando segurança e
eficiência.~\cite{ache2015}

\subsection{Titulação}\label{titulacao}

``Titulação é um procedimento no qual a quantidade de analito de uma amostra é determinada
adicionando-se uma quantidade conhecida de um reagente que reage completamente com o analito de uma
forma bem definida'' (HAGE, CARR. 2012)~\cite{Hage2012}.

Titulações volumétricas compreendem a medida de volume de uma solução padrão (de concentração
conhecida) necessária para reagir completamente com o analito.~\cite{Skoog2014}

Em uma titulação ácido-base, o titulado (analito) é um ácido e o titulante (solução ou composto com
os metros conhecidos) é uma base, ou vice-versa. Para saber quando e quanto de todo o analito
reagiu, é necessário adicionar um indicador ácido-base, uma substância química que muda de cor em
uma faixa conhecida de pH, ajudando a detectar o ponto estequiométrico ou ponto de equivalência.

``O ponto de equivalência é o ponto teórico alcançado quando a quantidade adicionada do titulante é
quimicamente equivalente à quantidade de analito na amostra''~\cite{Skoog2014}. O ponto final é o
ponto onde ocorre visualmente a percepção de alterações físicas (cor ou turbidez) pelo observador.

“Entre o ponto final da titulação e o ponto estequiométrico (teórico) sempre existirá uma pequena
diferença de volume do titulante chamada de Erro de Titulação”.~\cite{Ruy1999}

A reação de neutralização do AAS é dada abaixo:

\begin{center}
    \ce{C8O2H7COOH + NaOH -> C8O2H7COONa + H2O}
\end{center} 

Segundo Skoog et al. (2014), na titulação de um ácido fraco (HA) com uma base forte (\ce{NaOh} ou
\ce{KOH}) ocorrem as seguintes etapas:

\begin{enumerate}
    \item No início, a solução contém somente o analito, ácido fraco.
    \item Com a adição do titulante (até antes do ponto de equivalência), a solução contém uma série
        de tampões, entre a base conjugada formada da reação e o ácido fraco residual que permanece.
    \item No ponto de equivalência, a solução contém apenas o conjugado do ácido (um sal).
    \item Após o ponto de equivalência, o excesso de titulante básico reprime o caráter ácido ou
        alcalino do sal formado, produto da reação, sendo o pH resultante da concentração do excesso
        de titulante.
\end{enumerate}

\begin{figure}[H]
\begin{center}
    \includegraphics[scale=.9]{figuras/titulacao.jpg}
\end{center}
\caption{Curva de titulação Ácido fraco $\times$ Base forte}
\label{fig:curva_titulacao}
\end{figure}

