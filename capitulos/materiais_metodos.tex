     \chapter{Materiais e Métodos}\label{sec:mat}
     \section{Materiais}\label{sub:Materiais}
     
     \begin{tabular}{l l}
         \toprule
         \multicolumn{1}{c}{\textbf{Materiais}} & \multicolumn{1}{c}{\textbf{Reagentes}} \\
         \midrule
     \tabitem Almofariz com pistilo & \; \tabitem Água destilada \\
     \tabitem Balão volumétrico 1 L & \multirow{2}{*}{%
         \begin{tabular}{l}
     \tabitem Comprimidos:  \\
     Aspirina$^{\tiny{\textregistered}}$, 
     Melhoral, Genérico e AAS Infantil \end{tabular}}\\

     \tabitem Balança analítica & \\
     \tabitem Bastão de vidro & \; \tabitem Etanol comercial \\
     \tabitem Bureta & \; \tabitem Indicador fenolftaleína 0,1\% \\
     \tabitem Béqueres & \; \tabitem Solução padrão de \ce{NaOH} 0,100 mol/L \\
     \tabitem Conta gotas & \\
     \tabitem Erlenmeyer & \\
     \tabitem Espátulas  & \\
     \tabitem Pipeta volumétrica 10 mL  & \\
     \tabitem Pisseta  & \\
     \tabitem Provetas: 50, 100 e 500 mL  & \\
     \tabitem Suporte universal com garra e mufa  & \\
     \bottomrule
     \end{tabular}\hfill\

     \section{Métodos}\label{sub:metodos}

     \subsection{Experimento 1}\label{exp1}

     \subsubsection{Procedimento A}\label{ProcedimentoA}

     Diluiu-se uma solução de \ce{NaOH} 0,2 M afim de obter-se uma solução 0,1 M. Em uma proveta de 500 mL 
     ambientada, mediu-se 500 mL da solução 0,2 M, pipetando-se no final para maior precisão. 
     Transferiu-se quantitativamente o que estava na proveta para um balão volumétrico de 1 L, ou seja, 
     lavou-se as paredes da proveta três vezes com água destilada até garantir que toda a solução havia sido 
     transportada. Completou-se o volume do balão volumétrico até o menisco e homogeneizou-se a 
     solução, invertendo o recipiente tampado, verticalmente, várias vezes. Esta solução será o titulante.

     Foi preparada, com o uso de provetas, 300 mL de um solução 1:1 de etanol comercial e água destilada, 
     que seria o solvente utilizado para dissolver os comprimidos. 

     \subsubsection{Procedimento B}\label{ProcedimentoB}

     Um comprimido de cada medicamento foi pesado em balança analítica e os valores foram anotados. Os 
     comprimidos foram macerados em almofariz com o uso de pistilo e da massa de cada um foi pesada,
     em placa de Petri, uma amostra de 0,100g. Essa amostra foi dissolvida em 50 mL de solvente, em um béquer
     com agitação até a solubilização e posteriormente dividida em quatro amostras de 10 mL, que foram 
     transferidas para erlenmeyers utilizando-se pipeta volumétrica. Foram adicionadas duas gotas de 
     fenolftaleína a cada uma das quatro amostras, as quais serão os titulados.

     \subsubsection{Procedimento C}\label{ProcedimentoC}

     Transferiu-se o titulante para a bureta de 50 mL até zerá-la e deu-se início à titulação de ácido 
     acetilsalicílico com a solução de \ce{NaOH}. O processo de gotejamento do titulante permaneceu até 
     o aparecimento de uma coloração rósea no titulado que persistisse. Anotou-se o volume de titulante gasto 
     para cálculos posteriores. Esse processo foi repetido nas quatro amostras de cada um dos comprimidos.

     \subsection{Experimento 2}\label{exp2}

     Devido a inconsistências encontradas nos cálculos do experimento 1 (\ref{exp1}), foi decidido realizar 
     mais um experimento com algumas ligeiras mudanças nos procedimentos.
     
     \subsubsection{Procedimento A}\label{ProcedimentoA2}

     Para preparação do titulante (solução de \ce{NaOH} 0,1 M), prosseguiu-se da mesma forma que no procedimento
     \ref{ProcedimentoA}. Já o solvente, foi preparado de maneira similar, porém, em maior quantidade.

     \subsubsection{Procedimento B}\label{ProcedimentoB2}

     Como em \ref{ProcedimentoB}, um comprimido de cada medicamento foi pesado e macerado. Os comprimidos 
     macerados foram divididos em três amostras de 0,1000g e cada amostra foi dissolvida em 40 mL de 
     solvente, que foi transferido para um erlenmeyer, utilizando proveta, com agitação até solubilização. 
     Foram necessários dois comprimidos de AAS Infantil para obtenção das três amostras.
     Foram adicionadas três gotas de fenolftaleína a cada uma das três amostras, as quais serão os titulados.

     \subsubsection{Procedimento C}\label{ProcedimentoC2}

     Foi feita a titulação do ácido acetilsalicílico com a solução de \ce{NaOH} como especificado 
     em \ref{ProcedimentoC}. Esse processo foi repetido nas três amostras de cada um dos comprimidos.
