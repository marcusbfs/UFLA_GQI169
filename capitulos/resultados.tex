\chapter{Resultados}\label{resultadosk}

Equações utilizadas

\begin{equation}\label{ncv}
     n = \frac{C}{V} 
\end{equation}
onde
\begin{itemize}
    \item[$n$ :] número de mols
    \item[$C$ :] Concentração molar, mol/L
    \item[$V$ :] Volume, L
\end{itemize}

\begin{equation}\label{massamolar}
    MM = \frac{m}{n} 
\end{equation}
onde
\begin{itemize}
    \item[$MM$ :] Massa molar, g/mol
    \item[$m$ :] massa, g
    \item[$n$ :] número de mols
\end{itemize}


\bigskip
Dados gerais

\begin{itemize}
     \item Massa molar de ácido acetilsalicílico: 180,157 g/mol
     \item Concentração da solução \ce{NaOH}(titulante): 0,1 mol/L
     \item Volume de titulante (bureta): 50 mL
     \item Quantidade de NaOH (bureta): 2 mol, obtido da equação \eqref{ncv}
\end{itemize}

\section{Experimento 1}\label{res_exp1}

% Tabela pesagem
\begin{table}[H]\label{t:peso_1}
    \centering
    \begin{tabular}{l c c}
       \toprule
       Medicamento & $m_{comprimido}$ (g) & $m_{amostra}$ (g) \\
       \midrule
       Aspirina\R & 0,600 & 0,099  \\
       Genérico & 0,610 & 0,100  \\
       Melhoral\R & 0,587 & 0,101 \\
       AAS Infantil & 0,164 & 0,100 \\
        \bottomrule
    \end{tabular}
    \caption{Informações das pesagens no primeiro experimento}
\end{table}

% Tabela de dados titulados
\begin{table}[H]\label{titulacao_exp1}
    \centering
    \begin{tabular}{l c c c c c}

        \toprule
        Medicamento & $v_1$  & $v_2$ & $v_3$ & $v_4$ & $v_{\textrm{méd}}$ \\
        \midrule
        Aspirina\R     & 1,0 & 1,1 & 1,2 & 1,1 & 1,100 \\
        Genérico     & 1,0 & 1,0 & 1,0 & 0,9 & 0,975  \\
        Melhoral\R     & 1,0 & 1,0 & 1,0 & 1,1 & 1,025\\
        AAS Infantil & 0,7 & 1,0 & 0,8 & 0,7 & 0,800\\
       \bottomrule

    \end{tabular}
    \caption{Volumes gastos na titulação, em mL}
\end{table}

\begin{table}[H]\label{res_calculados_exp1}
    \centering
    \begin{tabular}{l r c c c }
        \toprule
        Medicamento & $n_{\textrm{AAS}}$ &$m_{AAS}$(titulado)& 
        $m_{\textrm{AAS}}$(exp) & $m_{\textrm{AAS}}$ (bula) \\

        \midrule
        Aspirina\R     & $1,1\cdot 10^{-4}$   &19,817 & 600,52 & 500\\
        Genérico     & $9,75\cdot 10^{-5}$  & 17,565 & 535,74 & 500 \\
        Melhoral\R     & $1,025\cdot 10^{-4}$ & 18,466 & 536,61 & 500 \\
        AAS Infantil & $8,0\cdot 10^{-5}$   & 14,413 & 118,18 & 100 \\
        \bottomrule
    \end{tabular}
    \caption{Resultados calculados}
\end{table}

onde
\begin{itemize}
    \item[] $n_{AAS}$: número de mol de AAS em uma alíquota  de 10 mL da solução 
        onde foi dissolvida a amostra. Obtido com a equação \eqref{ncv}
    \item[] $m_{AAS}$ (titulado): massa de AAS em uma alíquota de 10 mL da solução
        onde foi dissolvida a amostra, em mg.
        Obtida da equação \eqref{massamolar}
    \item[]$m_{AAS}$ (exp): massa calculada de AAS na massa total do comprimido, em mg
    \item[] $m_{AAS}$ (bula): massa de AAS no comprimido informada pela bula, em mg
\end{itemize}


\begin{table}[H]\label{porcentagem1}
    \centering
    \begin{tabular}{l c c}
        \toprule
        Medicamento & Experimental & Bula \\
        \midrule
        Aspirina\R     & 100,087 & 83,333 \\
        Genérico     & 87,826  & 81,967 \\
        Melhoral\R     & 91,416  & 85,179 \\
        AAS Infantil & 72,061  & 60,976 \\
        \bottomrule
    \end{tabular}
    \caption{Massa de AAS por massa do comprimido, em porcentagem}
\end{table}

\section{Experimento 2}\label{res_exp2}

\begin{table}[H]\label{t:peso_2}
    \centering
    \begin{tabular}{l c c c c}
       \toprule
       Medicamento & $m_{comprimido}$ (g) &$m_{amostra \, 1}$ (g) &
       $m_{amostra \, 2}$ (g) & $m_{amostra\, 3}$ (g)\\
       \midrule
       Aspirina\R     & 0,6029 & 0,0996 & 0,0998 & 0,0997  \\
       Genérico     & 0,6068 & 0,0996 & 0,0999 & 0,1003  \\
       Melhoral\R     & 0,6191 & 0,1000 & 0,0995 & 0,0998 \\
       AAS Infantil & 0,1622 & 0,0999 & 0,1000 & 0,1000 \\
        \bottomrule
    \end{tabular}
    \caption{Informações das pesagens no segundo experimento}
\end{table}

\begin{table}[H]\label{titulacao_exp2}
    \centering
    \begin{tabular}{l c c c c}

        \toprule
        Medicamento & $v_1$  & $v_2$ & $v_3$ &  $v_{\textrm{méd}}$ \\
        \midrule
        Aspirina\R   & 4,4 & 4,3 & 4,4 & 4,367 \\
        Genérico     & 4,2 & 4,3 & 4,3 & 4,267 \\
        Melhoral\R     & 4,2 & 4,3 & 4,3 & 4,267 \\
        AAS Infantil & 3,3 & 3,3 & 3,3 & 3,300 \\
       \bottomrule

    \end{tabular}
    \caption{Volumes gastos na titulação, em mL}
\end{table}

\begin{table}[H]
    \centering
    \begin{tabular}{c c c c}
        \toprule
        Amostra & $n_{AAS}$ & $m_{AAS}$ (titulado) & $m_{AAS}$ (exp) \\
        \midrule
        A1 & $4,4\cdot 10^{-4}$ & 79,269 & 479,833 \\
        A2 & $4,3\cdot 10^{-4}$ & 77,468 & 467,988\\
        A3 & $4,4\cdot 10^{-4}$ & 79,269 & 478,635 \\
        \bottomrule
    \end{tabular}
    \caption{Resultado da titulação da Aspirina\R}
    \label{res_aspirina}
\end{table}

\begin{itemize}
    \item $m_{AAS,\; med}$ (exp) $= 475,485$ mg
\end{itemize}

\begin{table}[H]
    \centering
    \begin{tabular}{c c c c}
        \toprule
        Amostra & $n_{AAS}$ & $m_{AAS}$ (titulado) & $m_{AAS}$ (exp) \\
        \midrule
        G1 & $4,2\cdot 10^{-4}$ & 75,666 & 460,985 \\
        G2 & $4,3\cdot 10^{-4}$ & 77,468 & 470,543 \\
        G3 & $4,3\cdot 10^{-4}$ & 77,468 & 468,667 \\
        \bottomrule
    \end{tabular}
    \caption{Resultado da titulação do genérico}
    \label{res_generico}
\end{table}

\begin{itemize}
    \item $m_{AAS,\; med}$ (exp) $= 466,732$ mg
\end{itemize}

\begin{table}[H]
    \centering
    \begin{tabular}{c c c c}
        \toprule
        Amostra & $n_{AAS}$ & $m_{AAS}$ (titulado) & $m_{AAS}$ (exp) \\
        \midrule
        M1 & $4,2\cdot 10^{-4}$ & 75,666 & 468,448 \\
        M2 & $4,3\cdot 10^{-4}$ & 77,468 & 482,011 \\
        M3 & $4,3\cdot 10^{-4}$ & 77,468 & 480,562 \\
        \bottomrule
    \end{tabular}
    \caption{Resultado da titulação do Melhoral\R}
    \label{res_melhoral}
\end{table}

\begin{itemize}
    \item $m_{AAS,\; med}$ (exp) $= 477,007$ mg
\end{itemize}

\begin{table}[H]
    \centering
    \begin{tabular}{c c c c}
        \toprule
        Amostra & $n_{AAS}$ & $m_{AAS}$ (titulado) & $m_{AAS}$ (exp) \\
        \midrule
        I1 & $3,3\cdot 10^{-4}$ & 59,452 & 96,527 \\
        I2 & $3,3\cdot 10^{-4}$ & 59,452 & 96,431 \\
        I3 & $3,3\cdot 10^{-4}$ & 59,452 & 96,431 \\
        \bottomrule
    \end{tabular}
    \caption{Resultado da titulação do AAS Infantil}
    \label{res_infantil}
\end{table}

\begin{itemize}
    \item $m_{AAS,\; med}$ (exp) $= 96,463$ mg
\end{itemize}

Das tabelas~\ref{res_aspirina} a~\ref{res_infantil}

\begin{itemize}
    \item[] $n_{AAS}$: número de mol de AAS em uma amostra dissolvida em 40 mL de solvente. Obtido
        com a equação \eqref{ncv}
    \item[] $m_{AAS}$ (titulado): massa de AAS em uma amostra dissolvida 
        em 40 mL de solvente, em mg. Obtida da equação \eqref{massamolar}
    \item[]$m_{AAS}$ (exp): massa calculada de AAS na massa total do comprimido, em mg
\end{itemize}

\begin{table}[H]\label{porcentagem2}
    \centering
    \begin{tabular}{l c c}
        \toprule
        Medicamento & Experimental & Bula \\
        \midrule
        Aspirina\R   & 78,866 & 82,932 \\
        Genérico     & 76,917  & 82,399 \\
        Melhoral\R     & 77,049  & 80,762 \\
        AAS Infantil & 59,472  & 61,652 \\
        \bottomrule
    \end{tabular}
    \caption{Massa de AAS por massa do comprimido, em porcentagem}
\end{table}
